% Options for packages loaded elsewhere
\PassOptionsToPackage{unicode}{hyperref}
\PassOptionsToPackage{hyphens}{url}
%
\documentclass[
]{article}
\usepackage{lmodern}
\usepackage{amssymb,amsmath}
\usepackage{ifxetex,ifluatex}
\ifnum 0\ifxetex 1\fi\ifluatex 1\fi=0 % if pdftex
  \usepackage[T1]{fontenc}
  \usepackage[utf8]{inputenc}
  \usepackage{textcomp} % provide euro and other symbols
\else % if luatex or xetex
  \usepackage{unicode-math}
  \defaultfontfeatures{Scale=MatchLowercase}
  \defaultfontfeatures[\rmfamily]{Ligatures=TeX,Scale=1}
\fi
% Use upquote if available, for straight quotes in verbatim environments
\IfFileExists{upquote.sty}{\usepackage{upquote}}{}
\IfFileExists{microtype.sty}{% use microtype if available
  \usepackage[]{microtype}
  \UseMicrotypeSet[protrusion]{basicmath} % disable protrusion for tt fonts
}{}
\makeatletter
\@ifundefined{KOMAClassName}{% if non-KOMA class
  \IfFileExists{parskip.sty}{%
    \usepackage{parskip}
  }{% else
    \setlength{\parindent}{0pt}
    \setlength{\parskip}{6pt plus 2pt minus 1pt}}
}{% if KOMA class
  \KOMAoptions{parskip=half}}
\makeatother
\usepackage{xcolor}
\IfFileExists{xurl.sty}{\usepackage{xurl}}{} % add URL line breaks if available
\IfFileExists{bookmark.sty}{\usepackage{bookmark}}{\usepackage{hyperref}}
\hypersetup{
  pdftitle={Practica Final Parte 1},
  pdfauthor={Xavier Vives, Josep Damià Ruiz, Antonio Marin y Martí Paredes},
  hidelinks,
  pdfcreator={LaTeX via pandoc}}
\urlstyle{same} % disable monospaced font for URLs
\usepackage[margin=1in]{geometry}
\usepackage{color}
\usepackage{fancyvrb}
\newcommand{\VerbBar}{|}
\newcommand{\VERB}{\Verb[commandchars=\\\{\}]}
\DefineVerbatimEnvironment{Highlighting}{Verbatim}{commandchars=\\\{\}}
% Add ',fontsize=\small' for more characters per line
\usepackage{framed}
\definecolor{shadecolor}{RGB}{248,248,248}
\newenvironment{Shaded}{\begin{snugshade}}{\end{snugshade}}
\newcommand{\AlertTok}[1]{\textcolor[rgb]{0.94,0.16,0.16}{#1}}
\newcommand{\AnnotationTok}[1]{\textcolor[rgb]{0.56,0.35,0.01}{\textbf{\textit{#1}}}}
\newcommand{\AttributeTok}[1]{\textcolor[rgb]{0.77,0.63,0.00}{#1}}
\newcommand{\BaseNTok}[1]{\textcolor[rgb]{0.00,0.00,0.81}{#1}}
\newcommand{\BuiltInTok}[1]{#1}
\newcommand{\CharTok}[1]{\textcolor[rgb]{0.31,0.60,0.02}{#1}}
\newcommand{\CommentTok}[1]{\textcolor[rgb]{0.56,0.35,0.01}{\textit{#1}}}
\newcommand{\CommentVarTok}[1]{\textcolor[rgb]{0.56,0.35,0.01}{\textbf{\textit{#1}}}}
\newcommand{\ConstantTok}[1]{\textcolor[rgb]{0.00,0.00,0.00}{#1}}
\newcommand{\ControlFlowTok}[1]{\textcolor[rgb]{0.13,0.29,0.53}{\textbf{#1}}}
\newcommand{\DataTypeTok}[1]{\textcolor[rgb]{0.13,0.29,0.53}{#1}}
\newcommand{\DecValTok}[1]{\textcolor[rgb]{0.00,0.00,0.81}{#1}}
\newcommand{\DocumentationTok}[1]{\textcolor[rgb]{0.56,0.35,0.01}{\textbf{\textit{#1}}}}
\newcommand{\ErrorTok}[1]{\textcolor[rgb]{0.64,0.00,0.00}{\textbf{#1}}}
\newcommand{\ExtensionTok}[1]{#1}
\newcommand{\FloatTok}[1]{\textcolor[rgb]{0.00,0.00,0.81}{#1}}
\newcommand{\FunctionTok}[1]{\textcolor[rgb]{0.00,0.00,0.00}{#1}}
\newcommand{\ImportTok}[1]{#1}
\newcommand{\InformationTok}[1]{\textcolor[rgb]{0.56,0.35,0.01}{\textbf{\textit{#1}}}}
\newcommand{\KeywordTok}[1]{\textcolor[rgb]{0.13,0.29,0.53}{\textbf{#1}}}
\newcommand{\NormalTok}[1]{#1}
\newcommand{\OperatorTok}[1]{\textcolor[rgb]{0.81,0.36,0.00}{\textbf{#1}}}
\newcommand{\OtherTok}[1]{\textcolor[rgb]{0.56,0.35,0.01}{#1}}
\newcommand{\PreprocessorTok}[1]{\textcolor[rgb]{0.56,0.35,0.01}{\textit{#1}}}
\newcommand{\RegionMarkerTok}[1]{#1}
\newcommand{\SpecialCharTok}[1]{\textcolor[rgb]{0.00,0.00,0.00}{#1}}
\newcommand{\SpecialStringTok}[1]{\textcolor[rgb]{0.31,0.60,0.02}{#1}}
\newcommand{\StringTok}[1]{\textcolor[rgb]{0.31,0.60,0.02}{#1}}
\newcommand{\VariableTok}[1]{\textcolor[rgb]{0.00,0.00,0.00}{#1}}
\newcommand{\VerbatimStringTok}[1]{\textcolor[rgb]{0.31,0.60,0.02}{#1}}
\newcommand{\WarningTok}[1]{\textcolor[rgb]{0.56,0.35,0.01}{\textbf{\textit{#1}}}}
\usepackage{graphicx,grffile}
\makeatletter
\def\maxwidth{\ifdim\Gin@nat@width>\linewidth\linewidth\else\Gin@nat@width\fi}
\def\maxheight{\ifdim\Gin@nat@height>\textheight\textheight\else\Gin@nat@height\fi}
\makeatother
% Scale images if necessary, so that they will not overflow the page
% margins by default, and it is still possible to overwrite the defaults
% using explicit options in \includegraphics[width, height, ...]{}
\setkeys{Gin}{width=\maxwidth,height=\maxheight,keepaspectratio}
% Set default figure placement to htbp
\makeatletter
\def\fps@figure{htbp}
\makeatother
\setlength{\emergencystretch}{3em} % prevent overfull lines
\providecommand{\tightlist}{%
  \setlength{\itemsep}{0pt}\setlength{\parskip}{0pt}}
\setcounter{secnumdepth}{-\maxdimen} % remove section numbering

\title{Practica Final Parte 1}
\author{Xavier Vives, Josep Damià Ruiz, Antonio Marin y Martí Paredes}
\date{}

\begin{document}
\maketitle

\tableofcontents 
\newpage

\hypertarget{cargad-en-un-dataframe-los-datos-del-fichero-listings.csv-y-construid-un-nuevo-data-frame}{%
\subsection{\texorpdfstring{\textbf{1. Cargad en un dataframe los datos
del fichero listings.csv y construid un nuevo data
frame}}{1. Cargad en un dataframe los datos del fichero listings.csv y construid un nuevo data frame}}\label{cargad-en-un-dataframe-los-datos-del-fichero-listings.csv-y-construid-un-nuevo-data-frame}}

\begin{Shaded}
\begin{Highlighting}[]
\NormalTok{datos_raw =}\StringTok{ }\KeywordTok{read.csv}\NormalTok{(}\StringTok{"listings.csv"}\NormalTok{)}
\NormalTok{datos =}\StringTok{ }\NormalTok{datos_raw[}\OperatorTok{-}\KeywordTok{c}\NormalTok{(}\DecValTok{1}\NormalTok{,}\DecValTok{2}\NormalTok{,}\DecValTok{3}\NormalTok{,}\DecValTok{5}\NormalTok{,}\DecValTok{14}\NormalTok{,}\DecValTok{15}\NormalTok{)]}
\end{Highlighting}
\end{Shaded}

\hypertarget{renommbrar-las-variables-al-castellano.}{%
\subsection{\texorpdfstring{\textbf{2. Renommbrar las variables al
castellano.}}{2. Renommbrar las variables al castellano.}}\label{renommbrar-las-variables-al-castellano.}}

\begin{Shaded}
\begin{Highlighting}[]
\KeywordTok{names}\NormalTok{(datos)[}\KeywordTok{names}\NormalTok{(datos) }\OperatorTok{==}\StringTok{ "host_name"}\NormalTok{] <-}\StringTok{ "Nombre_propietario"}
\KeywordTok{names}\NormalTok{(datos)[}\KeywordTok{names}\NormalTok{(datos) }\OperatorTok{==}\StringTok{ "neighbourhood"}\NormalTok{] <-}\StringTok{ "Vecindario"}
\KeywordTok{names}\NormalTok{(datos)[}\KeywordTok{names}\NormalTok{(datos) }\OperatorTok{==}\StringTok{ "latitude"}\NormalTok{] <-}\StringTok{ "Lat."}
\KeywordTok{names}\NormalTok{(datos)[}\KeywordTok{names}\NormalTok{(datos) }\OperatorTok{==}\StringTok{ "longitude"}\NormalTok{] <-}\StringTok{ "Long."}
\KeywordTok{names}\NormalTok{(datos)[}\KeywordTok{names}\NormalTok{(datos) }\OperatorTok{==}\StringTok{ "room_type"}\NormalTok{] <-}\StringTok{ "Tipo_habitación"}
\StringTok{names(datos)[names(datos) == "}\NormalTok{price}\StringTok{"] <- "}\NormalTok{Precio}\StringTok{"}
\StringTok{names(datos)[names(datos) == "}\NormalTok{minimum_nights}\StringTok{"] <- "}\NormalTok{Min.noches}\StringTok{"}
\StringTok{names(datos)[names(datos) == "}\NormalTok{number_of_reviews}\StringTok{"] <- "}\NormalTok{N_reseñas}\StringTok{"}
\StringTok{names(datos)[names(datos) == "}\NormalTok{number_of_reviews_ltm}\StringTok{"] <- "}\NormalTok{Res.mes}\StringTok{"}
\StringTok{names(datos)[names(datos) == "}\NormalTok{availability_}\DecValTok{365}\StringTok{"] <- "}\NormalTok{Disponibilidad_año}\StringTok{"}

\StringTok{names(datos)}
\end{Highlighting}
\end{Shaded}

\begin{verbatim}
##  [1] "Nombre_propietario" "Vecindario"         "Lat."              
##  [4] "Long."              "Tipo_habitación"    "Precio"            
##  [7] "Min.noches"         "N_reseñas"          "Res.mes"           
## [10] "Disponibilidad_año"
\end{verbatim}

\hypertarget{calcular-muxednimo-muxe1ximo-media-varianza-cuartiles-y-mediana.}{%
\subsection{\texorpdfstring{\textbf{3. Calcular mínimo, máximo, media,
varianza, cuartiles y
mediana.}}{3. Calcular mínimo, máximo, media, varianza, cuartiles y mediana.}}\label{calcular-muxednimo-muxe1ximo-media-varianza-cuartiles-y-mediana.}}

\begin{Shaded}
\begin{Highlighting}[]
\NormalTok{minimo =}\StringTok{ }\KeywordTok{unname}\NormalTok{(}\KeywordTok{sapply}\NormalTok{(datos[}\KeywordTok{c}\NormalTok{(}\DecValTok{3}\NormalTok{,}\DecValTok{4}\NormalTok{,}\DecValTok{6}\NormalTok{,}\DecValTok{7}\NormalTok{,}\DecValTok{8}\NormalTok{,}\DecValTok{9}\NormalTok{)],}\DataTypeTok{FUN=}\NormalTok{min))}
\NormalTok{maximo =}\StringTok{ }\KeywordTok{unname}\NormalTok{(}\KeywordTok{sapply}\NormalTok{(datos[}\KeywordTok{c}\NormalTok{(}\DecValTok{3}\NormalTok{,}\DecValTok{4}\NormalTok{,}\DecValTok{6}\NormalTok{,}\DecValTok{7}\NormalTok{,}\DecValTok{8}\NormalTok{,}\DecValTok{9}\NormalTok{)],}\DataTypeTok{FUN=}\NormalTok{max))}
\NormalTok{media =}\StringTok{ }\KeywordTok{unname}\NormalTok{(}\KeywordTok{sapply}\NormalTok{(datos[}\KeywordTok{c}\NormalTok{(}\DecValTok{3}\NormalTok{,}\DecValTok{4}\NormalTok{,}\DecValTok{6}\NormalTok{,}\DecValTok{7}\NormalTok{,}\DecValTok{8}\NormalTok{,}\DecValTok{9}\NormalTok{)],}\DataTypeTok{FUN=}\NormalTok{mean))}
\NormalTok{varianza =}\StringTok{ }\KeywordTok{unname}\NormalTok{(}\KeywordTok{sapply}\NormalTok{(datos[}\KeywordTok{c}\NormalTok{(}\DecValTok{3}\NormalTok{,}\DecValTok{4}\NormalTok{,}\DecValTok{6}\NormalTok{,}\DecValTok{7}\NormalTok{,}\DecValTok{8}\NormalTok{,}\DecValTok{9}\NormalTok{)],}\DataTypeTok{FUN=}\NormalTok{var))}
\NormalTok{cuartiles =}\StringTok{ }\KeywordTok{sapply}\NormalTok{(datos[}\KeywordTok{c}\NormalTok{(}\DecValTok{3}\NormalTok{,}\DecValTok{4}\NormalTok{,}\DecValTok{6}\NormalTok{,}\DecValTok{7}\NormalTok{,}\DecValTok{8}\NormalTok{,}\DecValTok{9}\NormalTok{)],}\DataTypeTok{FUN=}\NormalTok{quantile)}

\KeywordTok{rownames}\NormalTok{(cuartiles)<-(}\KeywordTok{c}\NormalTok{(}\StringTok{"Cuartil_0"}\NormalTok{,}\StringTok{"Cuartil_1"}\NormalTok{,}\StringTok{"Mediana"}\NormalTok{,}\StringTok{"Cuartil_3"}\NormalTok{,}\StringTok{"Cuartil_4"}\NormalTok{))}

\NormalTok{datosEstadisticos=}\DecValTok{0}

\NormalTok{datosEstadisticos=minimo}
\NormalTok{datosEstadisticos<-}\KeywordTok{rbind}\NormalTok{(datosEstadisticos, maximo)}
\NormalTok{datosEstadisticos<-}\KeywordTok{rbind}\NormalTok{(datosEstadisticos, media)}
\NormalTok{datosEstadisticos<-}\KeywordTok{rbind}\NormalTok{(datosEstadisticos, varianza)}
\NormalTok{datosEstadisticos<-}\KeywordTok{rbind}\NormalTok{(datosEstadisticos,cuartiles)}

\NormalTok{datosEstadisticos<-}\KeywordTok{round}\NormalTok{(datosEstadisticos, }\DataTypeTok{digits=}\DecValTok{5}\NormalTok{)}
\NormalTok{datosEstadisticos}
\end{Highlighting}
\end{Shaded}

\begin{verbatim}
##                       Lat.    Long.      Precio Min.noches  N_reseñas  Res.mes
## datosEstadisticos 35.81330 14.19540     7.00000    1.00000    0.00000  0.00000
## maximo            36.08025 14.57799  9111.00000 1000.00000  406.00000 69.00000
## media             35.93682 14.43258    90.41663    4.15562   19.30796  2.08290
## varianza           0.00369  0.00980 74014.58644  373.03480 1214.36211 18.29808
## Cuartil_0         35.81330 14.19540     7.00000    1.00000    0.00000  0.00000
## Cuartil_1         35.89924 14.37231    35.00000    1.00000    0.00000  0.00000
## Mediana           35.91563 14.48403    59.00000    2.00000    4.00000  0.00000
## Cuartil_3         35.95292 14.49943    95.00000    3.00000   22.00000  2.00000
## Cuartil_4         36.08025 14.57799  9111.00000 1000.00000  406.00000 69.00000
\end{verbatim}

\end{document}
